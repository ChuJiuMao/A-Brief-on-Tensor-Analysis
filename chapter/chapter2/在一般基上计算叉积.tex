\section{在一般基上计算叉积}
有时,可以将$\bb{v}$的顶分量与底分量分别简记为$(\bb{v})^{i}$和$(\bb{v})^{j}$。用这种记号,我们有
\begin{equation}\label{equ:2.11}
    \bb{u}\times \bb{v}=\left( \bb{u}\times \bb{v} \right) _k\bb{g}^k
\end{equation}
为了计算$\bb{u}\times \bb{v}$的底分量,我们设$\bb{u}=u^{i}\bb{g}_i$,$\bb{v}=v^{j}\bb{v}_j$,可得
\begin{align}
	\left( \bb{u}\times \bb{v} \right) _k&=\left( \bb{u}\times \bb{v} \right) \cdot \bb{g}^k\nonumber\\
	&=u^iv^j\left( \bb{g}_i\times \bb{g}_j \right) \cdot \bb{g}_k\nonumber\\
	&=u^iv^j\epsilon _{ijk}\label{equ:2.12}
\end{align}
$3\times 3\times 3$的符号$\epsilon _{ijk}$被称作排列张量$\bb{P}$的底分量(见练习2.13)。让我们来探究一下它的性质。由式\eqref{equ:2.12}和式\eqref{equ:1.26}
\begin{equation*}
    \epsilon_{123}=(\bb{g}_{1}\times \bb{g}_{2})\cdot \bb{g}_{3}=J
\end{equation*}
其中,$J$表示Jacobi行列式。如果我们交换符号中的两个指标,例如,交换上式中的2和3
\begin{equation*}
    \epsilon _{132}=\left( \bb{g}_1\times \bb{g}_3 \right) \cdot \bb{g}_2=\left( \bb{g}_2\times \bb{g}_1 \right) \cdot \bb{g}_3=-\left( \bb{g}_1\times \bb{g}_2 \right) \cdot \bb{g}_3=-J
\end{equation*}
然后在此式中交换指标1和2
\begin{equation*}
    \epsilon _{231}=\left( \bb{g}_2\times \bb{g}_3 \right) \cdot \bb{g}_1=\left( \bb{g}_1\times \bb{g}_2 \right) \cdot \bb{g}_3=J
\end{equation*}
进一步地,若有两个及以上的指标相等,那么$\epsilon_{ijk}=0$,因为我们总可以通过交换相乘的顺序,使得三重标积的前一个括号是它与自己作叉积。

总结一下就是
\begin{equation}\label{equ:2.13}
    \epsilon _{ijk}=\left\{ \begin{matrix}
        \begin{array}{c}
        +J\\
        -J\\
        0\\
    \end{array}&		\begin{array}{l}
        \text{若}\left( i,j,k \right) \text{为}\left( 1,2,3 \right) \text{的偶排列}\\
        \text{若}\left( i,j,k \right) \text{为}\left( 1,2,3 \right) \text{的奇排列}\\
        \text{若两个及以上指标相等}\\
    \end{array}\\
    \end{matrix} \right.  
\end{equation}
因此,回到式\eqref{equ:2.11},我们有
\begin{equation}\label{equ:2.14}
    \bb{u}\times \bb{v}=\epsilon _{ijk}u^iv^j\bb{g}^k
\end{equation}

\begin{example}
    已知$\bb{u}=2\bb{g}_1-\bb{g}_2+4\bb{g}_3$,$\bb{v}=2\bb{g}_1+3\bb{g}_2-\bb{g}_3$,其中基底$\{\bb{g}_{i}\}$由例题2.1给出,计算$\bb{u}\times \bb{v}$的底分量。
\end{example}
\begin{solution}
    我们用式\eqref{equ:2.14}的形式将式\eqref{equ:2.12}当$k=1$时的展开式形式写出
    \begin{align*}
        \left( \bb{u}\times \bb{v} \right) _1=\epsilon _{ij1}u^iv^j&=\underset{0}{\underbrace{\epsilon _{111}}}u^1v^1+\underset{0}{\underbrace{\epsilon _{121}}}u^1v^2+\underset{0}{\underbrace{\epsilon _{131}}}u^1v^3\\
        &+\underset{0}{\underbrace{\epsilon _{211}}}u^2v^1+\underset{0}{\underbrace{\epsilon _{221}}}u^2v^2+\underset{J}{\underbrace{\epsilon _{231}}}u^2v^3\\
        &+\underset{0}{\underbrace{\epsilon _{311}}}u^3v^1+\underset{-J}{\underbrace{\epsilon _{321}}}u^3v^2+\underset{0}{\underbrace{\epsilon _{331}}}u^3v^3\\
        &=J\left( u^2v^3-u^3v^2 \right)
    \end{align*}
    $\bb{u}$和$\bb{v}$的顶分量已经给出,由例题2.1可知
    \begin{equation*}
        J=\left| \begin{matrix}
            1&		0&		-1\\
            -1&		1&		-2\\
            2&		1&		1\\
        \end{matrix} \right|=6
    \end{equation*}
    因此
    \begin{equation*}
        \left( \bb{u}\times \bb{v} \right) _1=6\left[ \left( -1 \right) \left( -1 \right) -\left( 4 \right) \left( 3 \right) \right] =-66
    \end{equation*}
    类似地,我们可以给出$k=2$和$k=3$时的结果
    \begin{align*}
        \left( \bb{u}\times \bb{v} \right) _2&=J\left( u^3v^1-u^1v^3 \right) =6\left[ \left( 4 \right) \left( 2 \right) -\left( 2 \right) \left( -1 \right) \right] =60\\
        \left( \bb{u}\times \bb{v} \right) _3&=J\left( u^1v^2-u^2v^1 \right) =6\left[ \left( 2 \right) \left( 3 \right) -\left( -1 \right) \left( 2 \right) \right] =48
    \end{align*}
\end{solution}

为了计算$\bb{u}\times \bb{v}$的顶分量,我们可以仿造上面的过程,不过是将底基矢换做顶基矢。我们可以得到
\begin{equation}\label{equ:2.15}
    \left( \boldsymbol{u}\times \boldsymbol{v} \right) ^k=\epsilon ^{ijk}u_iv_j
\end{equation}
其中,$u_{i}$和$v_{j}$分别是$\bb{u}$和$\bb{v}$的底分量,$\epsilon^{ijk}$排列张量的顶分量,其定义如下:
\begin{equation}\label{equ:2.16}
    \epsilon ^{ijk}=\left( \boldsymbol{g}^i\times \boldsymbol{g}^j \right) \cdot \boldsymbol{g}^k=\left\{ \begin{matrix}
        \begin{array}{c}
        +J^{-1}\\
        -J^{-1}\\
        0\\
    \end{array}&		\begin{array}{l}
        \text{若}\left( i,j,k \right) \text{为}\left( 1,2,3 \right) \text{的偶排列}\\
        \text{若}\left( i,j,k \right) \text{为}\left( 1,2,3 \right) \text{的奇排列}\\
        \text{若两个及以上指标相等}\\
    \end{array}\\
    \end{matrix} \right.  
\end{equation}
换言之,$\epsilon ^{ijk}=J^2\epsilon _{ijk}$。

下面是一个有趣的恒等式,它将排列张量的分量与Kronecker delta符号联系起来,其证明:可以在练习2.5完成:
\begin{equation}\label{equ:2.17}
    \epsilon ^{ijk}\epsilon _{pqr}=\left| \begin{matrix}
        \delta _{p}^{i}&		\delta _{q}^{i}&		\delta _{r}^{i}\\
        \delta _{p}^{j}&		\delta _{q}^{j}&		\delta _{r}^{j}\\
        \delta _{p}^{k}&		\delta _{q}^{k}&		\delta _{r}^{k}\\
    \end{matrix} \right|
\end{equation}
若我们展开行列式并令$r=k$,我们可以得到
\begin{equation}\label{equ:2.18}
    \epsilon ^{ijk}\epsilon _{pqk}=\delta _{p}^{i}\delta _{q}^{j}-\delta _{q}^{i}\delta _{p}^{j}
\end{equation}
它与三重矢积式\eqref{equ:1.22}密切相关。见练习2.7。