\section{求和约定}
Einstein发明了求和约定,这让张量分析的受欢迎程度大大增加。在式\eqref{equ:2.1}中我们注意到,求和的哑指标i是重复的。此外,它的范围,1到3,我们从讨论的背景中可以得知。因此,在不损失任何信息的情况下,我们可以放弃式\eqref{equ:2.1}中的求和符号,而简单地写成
\begin{equation}\label{equ:2.2}
    \bb{v}=v^i\bb{g}_i
\end{equation}
“哑指标”的“哑”是指式\eqref{equ:2.1}和式\eqref{equ:2.2}中的指标$i$可以用任何其他符号代替而不影响求和的结果。因此$v^i\bb{g}_i=v^j\bb{g}_j=v^k\bb{g}_k$,类此皆可。将一个哑指标替换成另一个哑指标是想成为一个精通指标计算的人所需学习的最重要的技巧之一。

不过,我们还附加了一条限制规则:求和约定仅在一个指标为上标,另一个指标为下标时生效。因此,$v^iv_i=v^1v_1+v^2v_2+v^3v_3$,而$v^iv^i=v^1v^1$或$v^2v^2$或$v^3v^3$。通常来讲,【重复的上标或下标仅出现在变量或其分量的量中】。(练习2.21中讨论的笛卡尔张量符号是本限制规则的一个例外。)