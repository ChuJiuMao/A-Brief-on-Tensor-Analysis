\section{简化点积在一般基上的分量形式}

让我们回到“简化$\bb{u}\cdot \bb{v}$的分量展开式”的问题。当我们用相同的基底写出式\eqref{equ:2.3}的展开式形式时,写出来会十分繁琐。但如果我们设$\bb{u}=u^i\bb{g}_i$而$\bb{v}=v_j\bb{g}^j$,那么
\begin{equation}\label{equ:2.9}
    \bb{u}\cdot \bb{v}=u^iv_j\bb{g}_i\cdot \bb{g}^j=u^iv_j\delta _{i}^{j}=u^1v_1+u^2v_2+u^3v_3
\end{equation}
这个过程可比引入$\bb{u}$和$\bb{v}$笛卡尔坐标简单多了。当然我们还有一种选择,设$\bb{u}=u_i\bb{g}^i$,$\bb{v}=v^j\bb{g}_j$,那么
\begin{equation}\label{equ:2.10}
    \bb{u}\cdot \bb{v}=u_iv^i
\end{equation}

\begin{example}
    给定$\bb{u}=2\bb{g}_1-\bb{g}_2+4\bb{g}_3$以及$\bb{w}=-3\bb{g}^1+2\bb{g}^2-2\bb{g}^3$,计算$\bb{u}\cdot \bb{v}$,$\bb{w}\cdot \bb{v}$和$\bb{u}\cdot \bb{w}$,其中$\bb{v}$分别采用例题2.3的底基矢量和例题2.1中的顶基矢量。
\end{example}
\begin{solution}
    $\bb{u}$的顶分量已知,$\bb{v}$的底分量由例题2.3给出。因此,根据式\eqref{equ:2.9}
    \begin{equation*}
        \bb{u}\cdot \bb{v}=\left( 2 \right) \left( 12 \right) +\left( -1 \right) \left( 9 \right) +\left( 4 \right) \left( -3 \right) 
    \end{equation*}
    $\bb{w}$的底分量已知,$\bb{v}$的顶分量由例题2.1给出。因此,根据式\eqref{equ:2.10}
    \begin{equation*}
        \boldsymbol{w}\cdot \boldsymbol{v}=\left( -3 \right) \left( 2 \right) +\left( 2 \right) \left( 3 \right) +\left( -2 \right) \left( -1 \right) =2
    \end{equation*}
    最终,根据式\eqref{equ:2.9}
    \begin{equation*}
        \boldsymbol{u}\cdot \boldsymbol{w}=\left( 2 \right) \left( -3 \right) +\left( -1 \right) \left( 2 \right) +\left( 4 \right) \left( -2 \right) =-16
    \end{equation*}
\end{solution}