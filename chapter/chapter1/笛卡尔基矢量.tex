\section{笛卡尔基矢量}
若已知任意一个矢量$\bb{v}$的分量$(v_x,v_y,v_z)$,根据式\eqref{equ:1.6}与式\eqref{equ:1.5},我们置
\begin{equation}\label{equ:1.15}
    \begin{aligned}
        \left( v_x,v_y,v_z \right) &=\left( v_x,0,0 \right) +\left( 0,v_y,0 \right) +\left( 0,0,v_z \right)\\
        &=v_x\left( 1,0,0 \right) +v_y\left( 0,1,0 \right) +v_z\left( 0,0,1 \right)
    \end{aligned}
\end{equation}
回想式\eqref{equ:1.2}和式\eqref{equ:1.6},我们可以推出$\bb{v}$有唯一的表达式
\begin{equation}\label{equ:1.16}
    \bb{v}=v_x\bf{e}_x+v_y\bf{e}_y+v_z\bf{e}_z
\end{equation}
矢量集${\bf{e}_x,\bf{e}_y,\bf{e}_z}$被称作\textbf{标准笛卡尔基},而它的元素即为笛卡尔基矢量。矢量$\bb{v}$的笛卡尔坐标也可以称作$\bb{v}$相对于${\bf{e}_x,\bf{e}_y,\bf{e}_z}$的分量。