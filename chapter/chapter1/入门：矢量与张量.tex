\extrainfo{“这个理论对所有真正懂它的人产生了深刻的影响;它象征着由Gauss、Riemann、Christoffel、Ricci和Levi-Civita创立的绝对微分学的胜利。”\footnote{Albert Einstein, "Contribution to the Theory of General Relativity", 1915; as quoted andtranslated by C. Lanczos in \href{https://aapt.scitation.org/doi/10.1119/1.9734}{The Einstein Decade}, p. 213.}}


这本小书讲的正是爱因斯坦的点金石---绝对微积分绝对微积分,如今它被称作\textbf{张量分析}。不过,我写这本书时并没有着眼于广义相对论,而是着眼于连续介质力学,这是一种更为简单的理论,它分析“我们每天看到和使用的大量物质:空气、水、土、人体、木头、石头、钢铁、混凝土、玻璃、橡胶……”\footnote{Truesdell and Noll , The Non-Linear Field Theories of Me chanics, p. 1. Two outstanding introductory texts on continuum mechanics are A First Course in Rational Continuum Mechanics, 2nd ed, by Truesdell and Continuum Mechanics by Chadwick.}

连续介质力学虽然是广义相对论的一种极限情况,但是我们在处理它的时候最好是根据它的特点来处理。这样来看的话,两种理论的基础具有本质差别。连续介质力学需用的几何是三维欧几里得空间(简称$E_3$)与一维实轴。而广义相对论需用的几何是四维黎曼流形(球面是二维黎曼流形)。如果你只满足于精通广义相对论(并想参考Misner、Thorne 和 Wheeler 的《万有引力》),看到这里也请振作。在我们即将打造的工具中,我们也可以找到攀登广义相对论顶峰的装备。而那些灌溉了,或者说嵌入,连续介质力学花园的东西本质上是一种弯曲的二维连续体,称作壳,它以不起眼的形式反映了在盛开的广义相对论的花朵中发现的几乎所有数学枝叶。

在试图用数学语言描述力学规律时,我们面临着两个问题。一个是,如果要量化物理现象和对象,就必须引入参考系和该参考系下的坐标系\footnote{所谓参考系,就是用数学的语言来描述物理世界,它给物理世界(\textbf{World},简记作$\mathscr{W} $)中每一个事件(\textbf{Event},简记作$e$)都打上了一个“戳”,这个“戳”的信息包括它在$E_3$中的位置(即空间点)以及它发生的时刻,用实轴$\mathbb{R} $上的一个点表示。换言之,参考系就是从物理世界$\mathscr{W} $到$E_3\times \mathbb{R} $的一个映射$f$。

参考系中的坐标系为每个位置分配一个唯一的三元实数组$n,v,w$,称为空间坐标,并为每个时刻分配一个唯一的数字$t$,称为时间.}。另一个是,由于参考系与坐标系仅仅是脚手架,所以我们无需参考系与坐标系也应当能(以不变的形式)表述物理定律。事实上,这就是广义相对论的宏伟目标。

然而,在连续介质力学中,存在着被称为惯性系的特殊参考系;牛顿的质点运动定律---力等于质量乘以加速度加速度---只在这种参考系下成立\footnote{惯性系也很特殊,但是是另一种不同,而在广义相对论中,参考系是坐标系! (物理学是几何学) 我们可以在广义相对论中引入惯性系,就像我们可以在球体上一点的任意小邻域中引入二维笛卡尔坐标系一样。}。因此,连续介质力学的一个基本关注点是,像牛顿定律这样的定律如何从一个参考系转换为另一个参考系\footnote{改变参考系意味着,如果两个参考系彼此相互运动,同一事件$e$会在两个参考系中投影出不同的$E_3$中的位置$P_1$和$P_2$,以及$\mathbb{R}$上的不同时刻$T_1$和$T_2$。}。除了练习4.24外我们将不分析参考系的转换,相反,我们将探讨一个固定的参考系中,当一个坐标系(例如笛卡尔坐标系)被另一个坐标系(如球坐标系)所取代时,研究对象或是物理定律的数学表达式如何变化。

下文中,我将假设你已经学过一些平面几何以及立体几何的内容,并且假设你已经了解了一些关于矢量代数以及微积分的知识。为了简洁起见,我在探讨时省略了一些细节和例子,你可以参考其他专门介绍矢量的书。同时我也着重强调了几个重点,特别是大部分书中都找不到的,关于矢量加法与分量表示的物理意义。每一章结尾的练习旨在对正文中的内容进行补充和拓展。





