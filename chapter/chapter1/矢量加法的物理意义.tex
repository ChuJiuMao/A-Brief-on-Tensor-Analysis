\section{矢量加法的物理意义}

这里有必要强调作为物理或运动学属性的反映的矢量加法与在式\eqref{equ:1.6}右侧所使用的矢量加法之间的差别。例如,我们设$\bb{v}$表示刚体绕定点的转动。根据右手法则,取$\bb{v}$的方向与大小分别代表转轴与转动角度。如果随后又发生了一次转动,用矢量$\bb{u}$表示,那么从运动学的角度考虑,这两次连续的转动得到的效果可以用一次转动予以替代,我们设为$\bb{w}$。然而,通常来讲$\bb{v}+\bb{u}\ne \bb{w}$,换言之,连续有限转动不能如矢量一般叠加\footnote{转动也可以用矩阵来表示。如果两个连续的旋转由矩阵$V$和$U$表示,则它们等效于由矩阵$W = UV$表示的单个旋转。$u$、$v$或$w$可以记作$U$、$V$或$W$的实特征向量。}。不过,如果我们不将矢量的求和解释为旋转的叠加,那么式\eqref{equ:1.6}右侧的加法就是有意义的。我可以试着用一个简单的比喻来说明这一点。假设一个教室里有20个学生和30把椅子。从统计学角度来讲,我们可以说 "每把椅子上有$\frac{2}{3}$个学生",尽管在现实生活中并不存在$\frac{2}{3}$个学生。这是因为我们用整数来代替了学生和椅子的数量,为了得到统计数据,我们就需要按照计算法则来计算这些整数。

比方说,当矢量$\bb{v}$如式\eqref{equ:1.16}中所示表示力的时候,矢量加法就可以反映物理性质。“力具有矢量叠加性”是一个实验事实。参考练习1.3。在这种情况下,数学意义上的矢量分解恰好有一个直接的物理解释。这里再举一个例子,一个餐厅有20瓶1加仑的牛奶和30个罐子,此时,我们不止可以从数学意义上讲“每个罐子有$\frac{2}{3}$加仑的牛奶”,在实际中,我们也确实可以让每个罐子盛$\frac{2}{3}$加仑的牛奶。这里,数学意义上的整数除法就有一个对应的物理意义。