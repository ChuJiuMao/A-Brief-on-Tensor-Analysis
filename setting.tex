\documentclass[oneside,heading = true,12pt,openany]{ctexbook}

%需要大段设置的宏包
\usepackage{hyperref}
\hypersetup{
    colorlinks=true,
    linkcolor=black}

\ctexset {
    section/numbering = false,
    chapter = {
        beforeskip = 0pt,
        fixskip = true,
        format = \zihao{1},
        titleformat = \zihao{0},
        name = {第,章},
        nameformat = \vspace{-0.4cm}\rule{\linewidth}{2bp}\kaishu\\,
        aftername = \vspace{-2.1cm}\par\medskip\rule{\linewidth}{2bp}\vspace{0.4cm}\kaishu\\,
        pagestyle = plain,
    },
    section = {
        format = \raggedright\zihao{2},
        name = {第,节},
        number = \chinese{section},
    }
}


\usepackage{geometry}
\geometry{
    paperwidth=7.5in, 
    paperheight=10in,
    margin=16mm,
    headheight=2.17cm,
    footskip=8mm
}

%无需或仅需少量设置的宏包
\usepackage{amsmath,amsfonts,graphicx,caption,mathtools,amssymb,enumerate,pdfpages,arydshln,makeidx,tikz,fontspec,fancyhdr}
\usepackage[mathscr]{eucal}
\usepackage[allcommands]{overarrows}
\NewOverArrowCommand{overrightharpoon}{%
end=\rightharpoonup
}

\pagestyle{fancy}

%新命令
\newcommand\bb[1]{\boldsymbol{#1}}%斜体加粗
\renewcommand\bf[1]{\mathbf{#1}}%正体加粗
\newcommand\rr[1]{\overrightharpoon{#1}}%半箭头
\newcommand\qed{\rightline{$\square$}}%证毕符号
\newcommand\extrainfo[1]{{\small\kaishu{#1}\vspace{0.4in}}}%章前谶言
\newcommand\degree{^{\circ}}%角度符号
\newcommand\tr{\mathrm{tr}~}%迹符号


%新环境

%例题环境
\newcounter{ExampleCounter}
\counterwithin*{ExampleCounter}{chapter}
\newenvironment{example}
	{\stepcounter{ExampleCounter}
	\begin{description}
		\item[例题~\thechapter.\theExampleCounter]
	}{\end{description}
	}

%解答环境
\newenvironment{solution}
	{\begin{description}
		\item[解:]
	}{\end{description}
	}

%章末练习环境
\newcounter{ExerciseCounter}
\counterwithin*{ExerciseCounter}{chapter}
\newenvironment{exercise}
    {\stepcounter{ExerciseCounter}
    \section{练习}
    \begin{enumerate}[\thechapter.1.]
    }{\end{enumerate}
    }

%证明:环境
\newenvironment{proof}
	{\begin{description}
		\item[证明:]
	}{
        
        \qed
        \end{description}
	}



